\chapter{Introduction}

An ornithopter is an aircraft that flies by mimicking the flapping wing motion of birds and insects. This project aims to design and develop a functional ornithopter, exploring the principles of bio-inspired flight to achieve sustained, controlled flight. The ornithopter will utilize lightweight materials, efficient aerodynamic structures, and a compact propulsion system to emulate the natural flapping wing mechanisms seen in nature. 

The project involves designing wings capable of generating both lift and thrust through flapping motion while maintaining structural integrity. The wings will be controlled by a mechanical system powered by either an electric motor or a rubber-band mechanism for smaller models. A comprehensive aerodynamic analysis, mechanical design, and material selection will be carried out to optimize the flight performance of the ornithopter. 

Furthermore, control mechanisms will be developed to manage flight stability and direction. Through this project, we aim to deepen the understanding of bio-inspired flight systems and potentially contribute to the development of more efficient aerial vehicles in the future. The final model will be tested for its flight endurance, stability, and maneuverability to assess its practical applications in surveillance, environmental monitoring, and exploration.