\chapter{Conclusion}

The ornithopter project has made significant progress, with the successful development of a functional prototype incorporating key electronics and mechanical systems. This section summarizes the current progress and provides an overall assessment of the project's status.

\section{Summary of Current Progress}
The following milestones have been achieved in the project:

\begin{itemize}
    \item The power system, consisting of a \lipo battery, \bldc motor, and \esc, has been successfully integrated, providing reliable energy management.
    
    \item The wing-flapping mechanism has been constructed using Berlin gears and servo motors to achieve controlled and efficient wing movement.
    
    \item Nylon cloth wings have been attached, ensuring that the ornithopter is lightweight while capable of generating lift through flapping.
    
    \item Initial tests have demonstrated that the ornithopter is capable of flapping motion, with the motor and gear systems functioning as expected.
\end{itemize}

\section{Overall Assessment of Project Status}
\begin{itemize}
    \item The project is currently in a well-advanced stage, with most core components of the ornithopter operational. The mechanical and electronics systems are fully integrated, and the wing-flapping mechanism has been fine-tuned for basic functionality. While the project has met its key objectives in terms of prototype development, further testing and optimization will be required to refine flight performance and increase flight stability.
    
    \item Challenges, such as power consumption optimization and gear alignment issues, have been successfully resolved, indicating a strong technical foundation for further improvements. The project remains on track, with the potential for additional advancements in control and aerodynamics.
    
    \item In conclusion, the ornithopter project has achieved a stable, functioning prototype that is ready for further testing and enhancement. The groundwork has been laid for continued development and optimization, with promising prospects for achieving sustained flight in the final iteration.
\end{itemize}