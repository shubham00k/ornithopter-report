\chapter{Design And Development}

\section{Overview of Design Specifications}
The design and development phase of the ornithopter project is ongoing, with several key specifications already completed. The current focus is on creating a lightweight, efficient, and aerodynamic ornithopter capable of sustained flight through flapping wing motion. The primary design goals include:

\begin{itemize}
    \item \textbf{Wing Design}: The wing structure is based on the aerodynamic profiles observed in bird flight, with a focus on generating sufficient lift during the downstroke and minimizing drag during the upstroke.
    
    \begin{figure}[H]
        \centering
        \includegraphics[width=0.6\textwidth]{wing_design}
        \caption{Wing Design}
        \label{fig:wing_design}
    \end{figure}
    
    \item \textbf{Gearbox}: The Gearbox design is done, utilizing lightweight materials such as carbon fiber to maintain structural integrity while reducing overall weight.
    
    \begin{figure}[H]
        \centering
        \includegraphics[width=0.6\textwidth]{gearbox}
        \caption{Gearbox}
        \label{fig:gearbox}
    \end{figure}
    
    \item \textbf{Flapping Mechanism}: A functional body with flapping mechanism is completed to convert rotational motion from the motor into a reciprocating motion, controlling the wing movement at variable speeds.
    
    \begin{figure}[H]
        \centering
        \includegraphics[width=0.6\textwidth]{flapping_mechanism}
        \caption{Flapping Mechanism}
        \label{fig:flapping_mechanism}
    \end{figure}
\end{itemize}

\section{Components Selected and Rationale}
Several key components have been selected for this project, with a focus on ensuring efficient and reliable performance:

\begin{itemize}
    \item \textbf{\bldc Motor}: A \textbf{Brushless DC (\bldc) motor} has been selected for its high efficiency, power-to-weight ratio, and reliability. This motor will drive the flapping mechanism.
    
    \item \textbf{Electronic Speed Controller (\esc)}: The \textbf{\esc} is chosen to regulate the speed of the motor, allowing for precise control over the flapping frequency. This will enable adjustments to the wing motion based on different flight conditions.
    
    \item \textbf{\lipo Battery}: A lightweight \textbf{Lithium Polymer (\lipo)} battery was selected to power the motor and control system. The high energy density of the \lipo battery ensures that the ornithopter can maintain flight for extended periods without adding excessive weight.
    
    \item \textbf{Servos}: Small, lightweight servos have been chosen to control the tail surfaces, ensuring precise maneuverability during flight. These servos are linked to the tail rudder and elevator for controlling pitch and yaw.
    
    \item \textbf{Flysky Radio Receiver}: A \textbf{Flysky Radio Receiver} is being used for manual control of the ornithopter, allowing adjustments to motor speed and flight direction during testing.
\end{itemize}

\section{Prototypes and Models Developed}
Although the project is still in the development phase, initial prototypes and design models have been created to test key components:

\begin{itemize}
    \item \textbf{Wing Prototype}: An initial wing prototype has been developed and tested for aerodynamic performance. Early tests focused on evaluating the lift generated by the wing shape and material flexibility during flapping.
    
    \item \textbf{Flapping Mechanism Prototype}: A basic prototype of the flapping mechanism was built to test the motion conversion from the \bldc motor. Adjustments are ongoing to refine the flapping frequency and the range of wing motion.
    
    \item \textbf{Frame Model}: A 3D model of the frame is under development, focusing on optimizing the layout to distribute weight efficiently while maintaining structural integrity.
\end{itemize}