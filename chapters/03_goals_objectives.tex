\chapter{Project Goals and Objectives}

\section{Initial Goals}
At the beginning of the project, the following goals were established to guide the design and development of the ornithopter:

\begin{itemize}
    \item \textbf{Design a Functional Ornithopter}: The primary goal is to design a bio-inspired ornithopter capable of achieving sustained, stable flight by using flapping wings to generate both lift and thrust.
    
    \item \textbf{Lightweight and Efficient Structure}: Utilize lightweight materials for the frame, wings, and power system to ensure high efficiency and minimize the overall weight, enabling the ornithopter to fly with minimal power consumption.
    
    \item \textbf{Aerodynamic Wing Design}: Develop an optimized wing design that mimics the aerodynamics of birds or insects, ensuring effective lift generation during the downstroke and minimal drag during the upstroke.
    
    \item \textbf{Flapping Mechanism}: Implement a reliable flapping mechanism, converting the motor's rotational motion into a reciprocating (flapping) motion, capable of operating at variable frequencies for different flight conditions.
    
    \item \textbf{Control System Integration}: Integrate a control system with a radio receiver to enable manual control of the ornithopter's direction and stability during flight.
    
    \item \textbf{Stable Flight Performance}: Achieve a balance between lift and thrust to ensure stable flight, allowing the ornithopter to ascend, descend, and turn while maintaining controlled flapping motion.
\end{itemize}

\section{Changes and Adjustments to Objectives}
As the project progressed, several adjustments were made to the initial goals based on testing and performance analysis:

\begin{itemize}
    \item \textbf{Wing Structure Adjustment}: During early testing, it was observed that the wing structure required additional flexibility to reduce drag during the upstroke. The wing spars were modified to use more flexible materials, allowing the wings to bend and retract during the upstroke, improving overall efficiency.
    
    \item \textbf{Gearbox Optimization}: The original flapping mechanism was adjusted to incorporate a more efficient gearbox, improving torque transmission and allowing the motor to operate more effectively at lower speeds, which extended battery life and improved flapping control.
    
    \item \textbf{Control System Enhancement}: Initial flight tests revealed the need for more precise control of the tail surfaces. Additional servos were installed to fine-tune the tail movements, enhancing flight stability and maneuverability during turns and altitude adjustments.
    
    \item \textbf{Battery and Motor Adjustments}: The power system was optimized after initial flights, with a higher-capacity \lipo battery and an upgraded \bldc motor added to improve flight duration and performance.
    
    \item \textbf{Flapping Frequency Adjustment}: Based on aerodynamic tests, the flapping frequency was modified to strike a better balance between lift generation and forward thrust, ensuring smoother, more sustained flight.
\end{itemize}

The iterative process of design, testing, and modification ensured that the ornithopter met the initial goals while also addressing unforeseen challenges and optimizing performance as the project evolved.