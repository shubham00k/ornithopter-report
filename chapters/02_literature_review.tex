\chapter{Literature Review and Survey}

\section{Literature Review}
Ornithopters, inspired by the flapping flight of birds, have been a subject of growing research due to their potential applications in areas such as surveillance and micro aerial vehicles (MAVs). Recent studies, including those by Jackowski (2009) and Pillai et al. (2022), have highlighted the advancements in mechanical design, wing dynamics, and control systems for these biomimetic machines. 

Jackowski's work on a large-scale autonomous ornithopter, the "Phoenix," showcased the feasibility of controlling flapping flight using a simple proportional-derivative (PD) controller to stabilize pitch, demonstrating the potential for more complex control systems in larger applications. His research also addressed the balance between payload capacity, crash survivability, and field repairability, providing valuable insights into scaling ornithopter designs for research purposes. 

On the other hand, Pillai et al. explored a solenoid-powered flapping mechanism for a surveillance ornithopter, which enhanced energy efficiency and reduced mechanical complexity by eliminating traditional gear systems. This innovative approach points to the potential for lightweight and stealthy designs ideal for military and reconnaissance applications. However, challenges such as maintaining stability during flight and improving energy efficiency remain critical areas for further development. 

Overall, these studies reflect significant progress in ornithopter technology, particularly in enhancing maneuverability, control, and energy use, but also indicate that further research is needed to overcome existing limitations in flight dynamics and stability.

\section{Literature Survey}
Literature Survey on Ornithopters Introduction Ornithopters, which mimic the flapping flight of birds, have been a subject of fascination for centuries. They represent an engineering challenge in replicating the complex motions and aerodynamics of bird flight. Recently, their applications have broadened, particularly in the field of Micro Aerial Vehicles (MAVs) and surveillance drones. This survey outlines existing literature on ornithopter design, focusing on mechanical structures, flapping mechanisms, control systems, and the advantages and challenges of these machines.

\subsection{Review of Key Literature}
\begin{enumerate}
    \item \textbf{Opensource Ornithopter Prototype (2024)}: This study outlines the development of a remotely controlled Arduino-powered ornithopter. It details the step-by-step process of creating the ornithopter from scratch, covering design elements such as wingspan, weight, and flapping frequency. The prototype, with a wingspan of 1200-1400 mm and a flapping rate of 5-7 Hz, serves as an experimental platform for hobbyists, demonstrating a practical approach to ornithopter construction and control using readily available components.
    
    \item \textbf{Design and Construction of an Autonomous Ornithopter (Jackowski, 2009)}: This thesis presents the design and construction of a large-scale ornithopter, the "Phoenix." The project aimed to build a flapping wing vehicle that could carry a 400-gram payload of sensors for control research. The ornithopter achieved steady flight under computer control and demonstrated that simple Proportional-Derivative (PD) controllers can stabilize pitch during flight. Jackowski's work focuses on solving problems related to vehicle dynamics and control, showing that larger ornithopters retain many of the same challenges as smaller MAVs.
    
    \item \textbf{Design of Surveillance Ornithopter with Solenoid Flapper (Pillai et al)}: This project designed a surveillance ornithopter with a solenoid-powered flapping mechanism. The focus was on energy efficiency and stealth, making it suitable for surveillance operations. The ornithopter mimicked the flight of a golden eagle, and computational analysis using Ansys was performed to optimize wing load and flapping frequency. The solenoid flapper mechanism directly generated linear motion for wing flapping, eliminating the need for gears, which reduced the overall weight.
    
    \item \textbf{Biomimicry and Control of Flapping Wing MAVs (Various Studies)}: Studies from multiple institutions such as the University of Maryland and the University of Arizona focus on the control systems for MAVs that use flapping wing designs. Research emphasizes improving the maneuverability of small flapping wing vehicles while optimizing for energy efficiency and payload capacity.
\end{enumerate}

\subsection{Common Challenges Identified}
\begin{enumerate}
    \item \textbf{Complex Aerodynamics}: The unsteady and complex aerodynamics of flapping wings pose significant challenges. Unlike fixed-wing aircraft, the lift and thrust forces are dynamic and vary throughout the wing stroke, making control difficult.
    
    \item \textbf{Power Efficiency}: Many studies aim to address the high power consumption of ornithopters, particularly for micro and small-scale applications like MAVs. The use of solenoids and other efficient actuators has been explored to minimize energy use.
    
    \item \textbf{Control Systems}: Stabilizing an ornithopter in flight, especially under autonomous control, is another area of research focus. Simple controllers like the PD controller have proven effective in stabilizing pitch, but more advanced control strategies are being developed to improve overall flight stability.
\end{enumerate}